%%%%%%%%%%%%%%%%%%%%%%%%
%
% $Autor: Wings $
% $Datum: 2020-07-24 09:05:07Z $
% $Pfad: GDV/Vortraege/latex - Ausarbeitung/Kapitel/Einleitung.tex $
% $Version: 4732 $
%
%%%%%%%%%%%%%%%%%%%%%%%%

\chapter{Bézier-Curves}

\section{Introduction}

Bézier curves are parameter curves that can be used to represent free-form curves. They are named after the French engineer Pierre Bézier, who developed them in the 1960s at Renault to design car body shapes. During the same period, French physicist and mathematician Paul de Casteljau developed these curves independently of Pierre Bézier at Citroën. Paul de Casteljau's results were available earlier, but they were not published. This is the reason why Pierre Bézier is the namesake of these parameter curves.\cite{Farin:2002}

\bigskip


The Bézier curves are the basis for computer-aided design of models. This is referred to either as Computer Aided Design (CAD) or, when the emphasis is more on a mathematical view, Computer Aided Geometric Design (CAGD). \cite{Babovsky:2011} Computer requires a mathematical description of shapes to represent them. The most suitable description method for this is the use of parametric curves and surfaces. Here Bézier curves play the central role, because they are the most numerically stable polynomial bases used in CAD/CAGD software.\cite{Farin:2002}

\bigskip


Typography on the computer also uses Bézier curves. There are two ways of representing type with the computer. The simplest way is to save each individual letter with a fixed resolution and size as a bitmap and copy it to the memory area of the screen as needed. These so-called bitmap fonts can be displayed quickly but require a lot of memory if the characters are to be available in different sizes. In this case, an extra bitmap must be created for each size.  The quality also decreases when these bitmap fonts are scaled in size and resolution. Vector fonts are an alternative. Their name is derived from the fact that they use curves defined in a two-dimensional vector space to represent the characters. The advantage here is that the characters displayed in this way can be scaled without loss of quality. Two standards have developed for vector fonts. One is the TrueType font and the other is the PostScript font.  The TrueType fonts use square Bézier curves while the PostScript fonts use cubic Bézier curves. The cubic Bézier curves have more control points which leads to a better quality.
 
 \bigskip
 

Another field of application is the control of machine tools. When moving to a corner, the axis must be braked to a standstill at the corner point and then accelerated again after direction correction. This procedure ensures that it is not possible to move at a constant speed, which has negative consequences for the cycle time and quality. A possible solution to this problem is to place a curve between the two sections instead of a sharp corner, which can be run at a constant speed. Bézier curves are suitable for this purpose because only two points and two tangents are needed to form them. In the case of a corner, the points as well as the tangents would lie on the sections that form the corner. The resulting error would be analytically controllable and would allow an adjustment of the curve tolerance.\cite{Sencer:2014}