% !TeX spellcheck = de_DE/GB

\chapter{Weitere Hardware}
In den nachfolgenden Kapiteln folgt eine Aufzählung und Erläuterung der wesentlichen, zusätzlich benötigten Hardware. Dazu zählen Aktoren zur Darstellung des Betriebszustandes, Schalter und weitere Komponenten, die für den Aufbau des Schrittmotor-Demonstrators nötig waren.  

\section{OLED-Display}
Zur Feststellung des aktuellen Betriebszustandes und zur Ausgabe des eingestellten Bewegungsstufe des Schlittens dient ein 1,3\ Zoll OLED-Display \emph{(Organic Light Emitting Diode)} . Bei OLED-Displays sind jeweils drei OLEDs für einen Pixel zuständig und benötigt keine Hinterleuchtungen oder LC-Zellen, dadurch ist eine scharfe Kontrastierung möglich. Die OLED-Displays zeichnen sich durch die Reaktionsschnelligkeit aus. Die Helligkeit eines Pixels wechselt in weniger als einer Mikrosekunde.\cite{DieterStotz.2019} Insgesamt ist das Modul $36 \ mm \times 34  \ mm \times36 \ mm$ groß mit einer Bildschirmdiagonalen von $1,3\ Zoll $. Das Display besteht aus $128 \times 64 \ $ weißen OLED-Bildpunkten. Durch die I2C-Kommunikation kann das Display mit dem Arduino verbunden werden. Zur Stromversorgung wird  3,3\ V Gleichstromspannung, bei einem Stromverbrauch von weniger als 11\ mA, benötigt. Mithilfe von 4 Steckbrückenkabel ist es mit dem Arduino verbunden, um es mit der nötigen Spannung zu versorgen und die I2C-Verbindung herzustellen (Verdrahtung detailliert in Kapitel \ref{MLLS} \nameref{MLLS}). Die zur I2C-Kommunikation nötige Adresse ist 0x3F. %oder 0x27 (je nach Chip)
Zusätzliche Spezifikationen des Displays sind: 
	\begin{itemize}
		\item Pixelgröße: $0.21 \ mm \times 0,21 \ mm $
		\item Pixelabstand: $0.23 \ mm \times 0,23 \ mm $
		\item Anzeigemodus: Passive Matrix
		\item Pixelfarbe: Weiß
	\end{itemize}
\cite{AZDelivery.24}


\section{Signalleuchte}
Zusätzlich zum OLED-Display ist eine SMD-LED-Signalleuchte zur Zustandserkennung verbaut. Sie soll Störungen und Fehler erkenntlich machen. SMD steht für \emph{Surface Mount Device} und bedeutet, dass die Signalleuchte für Oberflächenmontagen konzipiert ist. Die Leuchte ist dabei mit einem einfachen Stecksystem am Gehäuse befestigt. Dazu wird sie in einer 8\ mm Montagebohrung des Gehäuses gesteckt und über die 10\ mm Gehäuseblende der Signalleuchte fixiert. Die Leuchte hat eine 3\ mm im Durchmesser große LED und emittiert ein rotes Licht. Die Betriebsspannung liegt zwischen 2 bis 2,4\ V und der Betriebsstrom bei 20\ mA. Angeschlossen wird sie über vier Litzen am Arduino (verweis auf Schaltplan).\cite{Mentor.2024}

\section{Spannungswandler}
Für den Aufbau des Demonstrators sind mehrere Spannungswandler nötig. Folgende Wandler wurden in der Schaltung verwendet.
\subsection{Netzteil SNT RD 50A}
Dieser Wechselspannung-/Gleichspannungswandler (AC/DC-Wandler) wandelt die 230\ V Wechselspannung des Netzanschlusses in 12\ V Gleichspannung um. %%%Verwendung im Bezug auf den Schrittmotor noch ergänzen
Weitere Spezifikationen des Wandlers sind:
\begin{itemize}
	\item \textbf{Bauteilabmessung ($L\times B \times H$):} $99 \ mm \times 97 \ mm \times 36 \ mm$
	\item \textbf{Ausgangsstrom:} 2\ A
	\item \textbf{Leistung:} 54\ W
	\item \textbf{Wirkungsgrad:} 79\ \% 
\end{itemize} 
\cite{Meanwell.2019}

\subsection{Spannungswandler ASM1117}
Dieser Spannungswandler wandelt die 12\ V vom Netzteil SNT RD 50A in jeweils 3,3\ V und 5\ V um. Die 3,3\ V werden für den Arduino, Schrittmotorsteuerung, LC-Display und dem Drehwinkel-Encoder verwendet. Weitere Technische Daten des Wandlers sind:
\begin{itemize}
	\item \textbf{Bauteilabmessung ($L\times B \times H$):} $40 \ mm \times 40 \ mm \times 20 \ mm$
	\item \textbf{Ausgangsstrom:} 800\ mA
\end{itemize}
	\cite{AMS}
	
\section{Linearführung}
Zur Demonstration einer linearen Bewegung wird eine kompakte Linearführung verwendet. Die Führung ist 400\ mm lang und 12\ mm breit. Diese Linearführung wird vor allem in Fused Deposition Modeling (FDM) -Drucker verwendet, wo es auf hohe Präzision bei niedrigen Toleranzen ankommt. Deshalb eignet sich diese Führung auch gut für dieses Projekt, wo es nicht darum geht, große Lasten zu Bewegen, sondern möglichst genau die Schrittauflösung auf eine Millimeterskala zu übertragen.
%Von dem Kaufteil gibt es kein Datenblatt

\section{Drehwinkel-Encoder}
Der Demonstrator soll mehrere Programme fahren können, deswegen wurde ein Drehwinkel-Encoder der Steuerung hinzugefügt. Dieser wird über fünf Pins am Arduino angeschlossen. Durch drehen des Drehschalters werden nacheinander zwei Kontakte geschlossen oder geöffnet. Dieser dadurch entstehende Signalfluss, bestehend aus zwei um 90 Grad versetzte Sinus bzw. Cosinus Schwingungen werden ausgewertet. Daraus wird bestimmt, in welcher Richtung (im oder gegen Uhrzeigersinn) und wie weit (inkrementell) gedreht wurde. Mithilfe dieser Logik kann durch ein Menü eine Bewegungsstufe ausgewählt werden, die der Schrittmotor fahren soll.\cite{Basler.2016} Bei einer Drehung im Uhrzeigersinn wird im Menü eine Bewegungsstufe höher und bei einer Drehung gegen den Uhrzeigersinn eine Bewegungsstufe niedriger ausgewählt. Zusätzlich zum Drehwinkel-Encoder ist auch noch ein Schaltfunktion im Bauteil selbst integriert. Durch eindrücken des Encoders wird ein Taster betätigt, durch denn der eingestellte Wert bestätigt und an den Arduino zur weiteren Verarbeitung weitergegeben wird. Zur Besseren Handhabung des Drehwinkel-Encoders wurde noch ein Drehgriff angefertigt und auf dem Drehgeber montiert.
Weitere Details: \begin{itemize}
	\item \textbf{Abmessungen ($b \times l \times h$):} $18 \ mm \times 31 \ mm \times 30 \ mm$
	\item \textbf{Betriebsspannung:} 3,3 - 5\ V
	\cite{SimacElec.2019}
\end{itemize}
\section{Schrittmotorsteuerung}	
Für eine einfachere Bedienung des Schrittmotors wurde Schrittmotortreiber mit integriertem Übersetzer verbaut. Bei einer Ausgangskapazität von bis zu 35\ V und $\mp$ 2\ A ist der DEBO DRV A4988 für den Betrieb von bipolaren Schrittmotoren ausgelegt. Auf dem Treiber ist ein Stromregler mit fester Ausschaltzeit integriert, in dem zwischen einer langsamen oder gemischten Abklingmodi gewählt werden kann. Durch eine Impulseingabe wird der Motor um einen Mikroschritt angetrieben. Der Vorteil der Schrittmotorsteuerung liegt darin, dass keine Phasensequenztabellen, Hochfrequenz-Steuerleitungen oder komplexe Schnittstellen programmiert werden müssen. Im Schrittbetrieb wählt der A4988 automatisch den Abklingmodi, langsam oder gemischt. Im gemischten Modus wird im Abklingvorgang zwischen Schnellen und langsamen Modus gewechselt. Dabei führt der gemischte Modus zu einer Verringerung der hörbaren Motorgeräusche, einer höheren Schrittgenauigkeit und einer geringeren Verlustleistung.  
Weitere Details: \begin{itemize}
	\item \textbf{Abmessungen ($b \times l \times h$):} $5 \ mm \times 5 \ mm \times 0,9 \ mm$
	\item \textbf{Steuerspannung:} 3,3 - 5\ V
	\item \textbf{Maximale Ausgangskapazität:} bis 35\ V und $\mp$ 2\ A
	\item \textbf{Schutzschaltungen:}  Thermische Abschaltschaltung, Schutz vor Masseschluss, Schutz vor kurzgeschlossener Last, Überkreuzungsstromschutz
\cite{Allegro.2022}
\end{itemize}




