%%%%%%%%%%%%%%%%%%%%%%%%
%
% $Autor: Wings $
% $Datum: 2020-07-24 09:05:07Z $
% $Pfad: GDV/Vortraege/latex - Report/Contents/Python.tex $
% $Version: 4732 $
%
%%%%%%%%%%%%%%%%%%%%%%%%

\chapter{Darstellung von Python-Programmen}


Es ist möglich eine Teil in das Dokument zu integrieren, siehe \ref{Code:Python:File:HelloWorld}. Dies ist die eleganteste Methode und auch zu bevorzugen. In der Liste der Optionen können auch einzelne Zeilen und Zeilenbereich ausgewählt werden. Falls eine Datei integriert wird, so ist diese stets so aktuell wie das Dokument.


\begin{code}
  \lstinputlisting[language=python]{../Code/HelloWorld/Blink.py}    
    
  \caption[\glqq Hello World\grqq{} in Python -- Variante 1]{Das Programm ``Hallo World'' in Python für Mikrocontroller-Boards wird aus der Datei \FILE{Blink.py} eingefügt.}\label{Code:Python:File:HelloWorld}    
\end{code}    

It may also be useful to integrate program lines, see \ref{Code:Python:HelloWorld}.



\lstset{caption=Some Code}
\begin{code}
  \begin{lstlisting}[language=python]
# Hello World for microcontroller boards
import pyb

redLED = pyb.LED(1) # built-in red LED
greenLED = pyb.LED(2) # built-in green LED
blueLED = pyb.LED(3) # built-in blue LED
while True:
    # Turns on the red LED
    redLED.on()
    # Makes the script wait for 1 second (1000 miliseconds)
    pyb.delay(1000)
    # Turns off the red LED
    redLED.off()
    pyb.delay(1000)
    greenLED.on()
    pyb.delay(1000)
    greenLED.off()
    pyb.delay(1000)
    blueLED.on()
    pyb.delay(1000)
    blueLED.off()
    pyb.delay(1000)
\end{lstlisting}      

  \caption[\glqq Hello World\grqq{} in Python -- Variante 2]{Das Programm ``Hello World'' in Python für Mikrocontroller-Boards wurde direkt in die \LaTeX{}-Datei eingefügt.}\label{Code:Python:HelloWorld}    
\end{code}    

Die Einbindung von Programmen mit Hilfe von Bildern ist sinnfrei.