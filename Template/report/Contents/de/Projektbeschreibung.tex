% !TeX spellcheck = de_DE/GB


\chapter{Projektbeschreibung}

\section{Aufgabenstellung}

Die Aufgabe besteht darin, mithilfe eines Arduino Nano 33 BLE Sense Lite einen Demonstrator für einen Schrittmotor zu entwickeln. Mittels einer Konstruktion, die über einen Schrittmotor verfügt, welcher einen Riemen antreibt, soll ein Schlitten auf einer Linearführung verfahren werden. Es sollen unterschiedliche Bewegungscharaketeristiken demonstriert werden. Durch die Programmierung des Arduino, wird der Schrittmotor gesteuert und wird in die gewünschte Richtung und Position bewegt. Dies beinhaltet eine korrekte Programmierung, sowie die korrekte Ausgabe durch den Schrittmotor und eine Konstruktion, die mit allen Teilen funktionsfähig ist. Nachfolgend wird die Hardware beschrieben, sowie die Entwicklung der Software und die Konstruktion und Fertigung des Demonstrators.

\section{Herausforderungen}

Das Projekt lässt sich in 3 verschiedene Teilprojekte unterteilen, die zum Abschluss der Aufgabe führen. Zuerst ist da die Programmierung und die Arbeit mit der Arduino IDE. Außerdem die Konstruktion des Demonstrators und zuletzt die hierfür benötigte Auswahl der Hardaware-Teile. Ebenfalls wichtig ist der Umgang mit den elektronischen Bauteilen, dass diese nicht durch den elektrischen Strom beschädigt werden und das der Demonstrator für den Transport geeignet ist. 

\section{Lösungsansätze}

Durch die Ausarbeitung der Herausforderungen, können Lösungsansätze hierfür entwickelt werden. Indem vorab eine Skizze für die Konstruktion angefertigt wird, kann darauf basierend die Suche nach Teilen durchgeführt werden. Um den Demonstrator mit all seinen Teilen und seiner unhandlichen Größe trotzdem transportfähig zu konstruieren, werden die elektronischen Bauteile mit auf dem Alu-Profil montiert. Außerdem spielt es für die Programmierung eine entscheidende Rolle, wie die verschiedenen Stufen angefahren werden sollen. Hierfür wurde sich von vornherein überlegt, wie die Stufen auszuwählen sind und in welche Positionen und in welchen Geschwindigkeiten verfahren wird.




