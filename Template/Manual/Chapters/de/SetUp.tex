%%%%%%%%%%%%
%
% $Autor: Wings $
% $Datum: 2019-03-05 08:03:15Z $
% $Pfad: ListOfParts.tex $
% $Version: 4250 $
% !TeX spellcheck = en_GB/de_DE
% !TeX encoding = utf8
% !TeX root = manual 
% !TeX TXS-program:bibliography = txs:///biber
%
%%%%%%%%%%%%

\chapter{Inbetriebnahme}

\begin{itemize} \label{STU}
	
	\item \textbf{Elastische Einstellung des Riemens}: Überprüfen Sie Spannung des Riemens, indem Sie den \textbf{Schlitten} manuell bewegen. Wenn während der Bewegung Schwierigkeiten oder ungewöhnliche Geräusche auftreten, stellen Sie Mithilfe der \textbf{Einstellschrauben} und einem Sechskantschlüssel die Spannung ein, um ein reibungsloses Gleiten des \textbf{Schlittens} zu gewährleisten. 
	
	\item \textbf{Einschalten}: Mit dem Umschalten des \textbf{Power}-Schalters wird das Gerät ein- oder ausgeschaltet. Das \textbf{Display} zeigt die voreingestellte Bewegungsstufe. Die \textbf{Status-LED} zeigt den Status des Demonstrators. Im eingeschalteten Zustand leuchtet diese grün.
	
	\item \textbf{Auswahl der Bewegungsstufe}: Über den \textbf{Drehschalter} wird in den verschiedenen Bewegungsstufen umgeschaltet (siehe Kapitel \ref{bew} \nameref{bew}). Im \textbf{Display} wird die ausgewählte Bewegungsstufe angezeigt.
	
	\item \textbf{Starten}: Mit dem Betätigen des \textbf{Start-Knopfs} beginnt der Demonstrationsablauf. Die \textbf{Status-LED} wechselt ihre Farbe auf gelb und der Demonstrator fährt eine Kalibrierfahrt. Sobald die \textbf{Status-LED} auf blau wechselt, beginnt der Demonstrationsablauf.
	
	\item \textbf{Stoppen}: Der Demonstrationsablauf kann \textbf{nicht} manuell gestoppt werden. Nach Ablauf wechselt die \textbf{Status-LED} die Farbe auf Grün und ist bereit für den nächsten Ablauf.
\end{itemize}