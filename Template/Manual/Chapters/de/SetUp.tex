%%%%%%%%%%%%
%
% $Autor: Wings $
% $Datum: 2019-03-05 08:03:15Z $
% $Pfad: ListOfParts.tex $
% $Version: 4250 $
% !TeX spellcheck = en_GB/de_DE
% !TeX encoding = utf8
% !TeX root = manual 
% !TeX TXS-program:bibliography = txs:///biber
%
%%%%%%%%%%%%

\chapter{Inbetriebnahme}

\begin{itemize} \label{STU}
	
	\item \textbf{Elastische Einstellung des Riemens}: Überprüfen Sie Spannung des \textbf{Riemens}, indem Sie den \textbf{Schlitten} manuell bewegen. Wenn während der Bewegung Schwierigkeiten oder ungewöhnliche Geräusche auftreten, lösen Sie, mithilfe eines Sechskantschlüssels die zwei \textbf{Einstellschrauben} auf der Seite der Frontblende und die zwei \textbf{Einstellschrauben} auf der Seite der Heckblende entgegen den Uhrzeigersinn. Drücken Sie mit ihrer Handkraft den \textbf{Einstellschlitten} soweit von der Vorderblende aus gesehen nach rechts, bis der Riemen ausreichend gespannt ist. Währenddessen ziehen Sie die beiden Einstellschrauben auf der Seite Vorderblende im Uhrzeigersinn fest. Überprüfen Sie, ob ein reibungsloses Gleiten des \textbf{Schlittens} vorhanden und ob der Riemen gespannt ist. Im Anschluss ziehen Sie die \textbf{Einstellschrauben} auf der Seite der Rückblende im Uhrzeigersinn fest. 
	
	\item \textbf{Einschalten}: Mit dem Umschalten des \textbf{Power}-Schalters wird das Gerät ein- oder ausgeschaltet. Das \textbf{Display} zeigt die voreingestellte Bewegungsstufe. Die \textbf{Status-LED} zeigt den Status des Demonstrators. Im eingeschalteten Zustand leuchtet diese grün.
	
	\item \textbf{Display}:  Im eingeschalteten Zustand wird im \textbf{Display} die ausgewählte Bewegungsstufe angezeigt.

	
	\item \textbf{Auswahl der Bewegungsstufe}: Über den \textbf{Drehschalter} wird in den verschiedenen Bewegungsstufen umgeschaltet (siehe Kapitel \ref{bew} \nameref{bew}). Wird der Drehschalter im Uhrzeigersinn um einen Schritt gedreht, zeigt das textbf{Display} die nächst höhere Bewegungsstufe. Wird der Drehschalter entgegen den Uhrzeigersinn um einen Schritt gedreht, zeigt das textbf{Display} die nächst niedrigere Bewegungsstufe. Wird nicht weiter gedreht, ist die im textbf{Display} angezeigte Bewegungsstufe, die ausgewählte Bewegungsstufe.
	
	
	\item \textbf{Starten}: Mit dem Betätigen des \textbf{Start-Knopfs} beginnt der Demonstrationsablauf. Die \textbf{Status-LED} wechselt ihre Farbe auf gelb und der Demonstrator fährt eine Referenzfahrt. Sobald die \textbf{Status-LED} auf blau wechselt, beginnt der Demonstrationsablauf.
	
	\item \textbf{Stoppen}: Der Demonstrationsablauf kann \textbf{nicht} manuell gestoppt werden. Nach Ablauf wechselt die \textbf{Status-LED} die Farbe auf Grün und ist bereit für den nächsten Ablauf.
\end{itemize}