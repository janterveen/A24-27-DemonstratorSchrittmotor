%%%%%%%%%%%%
%
% $Autor: Theilmann $
% $Datum: 2024-19-04 12:25:11Z $
% $Pfad: Bewegungsablauf.tex $
% $Version: 1 $
% !TeX spellcheck = en_GB/de_DE
% !TeX encoding = utf8
% !TeX root = manual 
% !TeX TXS-program:bibliography = txs:///biber
%
%%%%%%%%%%%%

\chapter{Bewegungsablauf}

In Zehn verschiedenen Bewegungsstufen zeigt der \textbf{Demonstrator} die Möglichkeiten eines Schittmotors. Es werden in einer Bewegunsstufen verschiedene Bewegungscharakteristiken demonstriert. 

\begin{itemize}
\item \textbf{Stufen 1-5}
	\begin{itemize}
\item\textbf{Beschleunigung}: Der \textbf{Schlitten} beschleunigt mit der in der ausgewählten Beschleunigung. Die Geschwindigkeit des \textbf{Schlittens} erhöht sich. 
	
	\item\textbf{Konstante Geschwindigkeit}: Nach erreichen der ausgewählten Geschwindigkeit, bewegt sich der \textbf{Schlitten} mit konstanter Geschwindigkeit weiter.
		
		\item\textbf{Verzögerung}: Der \textbf{Schlitten} verzögert um den ausgewählten Wert. Die Geschwindigkeit des \textbf{Schlittens} nimmt ab.
			
			\item\textbf{Genaue Position}: Der \textbf{Anzeigepfeil} steht beim Stoppen des \textbf{Schlittens} auf die genau ausgewählte Position.
\end{itemize}
\end{itemize}
\begin{itemize}
	\item \textbf{Stufen 6-10}
	\begin{itemize}
	\item\textbf{Pendeln}: Der \textbf{Schlitten} pendelt mit der ausgewählten Beschleunigung und Verzögerung vom Umkehrpunkt zum Umkehrpunkt.
\end{itemize}
\end{itemize}			
	