


\documentclass[12pt,a4paper]{scrbook}
\include{General/Packages}
\include{General/commands}
\include{General/tikzdefs}
\usepackage{circuitikz}
\begin{document}


	
	\chapter{Schnelleinstieg für Demonstrator für den Schrittmotor}
	
	\section{Lieferumfang}
	
	\begin{itemize}
		\item Demonstrator Schrittmotor
		\item Netzkabel
	\end{itemize}
	
	\section{Gerät aufstellen:}
	\begin{itemize}
		\item Stellen Sie den Demonstrator auf eine ebene, feste Oberfläche.
		\item Vermeiden Sie direkte Sonneneinstrahlung sowie Wärme- oder Feuchtigkeitsquellen.
		\item Keine Gegenstände auf den Demonstrator stellen.
		\item Halten Sie den Arbeitsbereich des Demonstrators frei von Gegenständen.
	\end{itemize}
	
	\section{Inbetriebnahme}
	
	\begin{enumerate}
		\item \textbf{Riemen einstellen:}
		\begin{itemize}
			\item Spannung überprüfen indem Sie den Schlitten manuell bewegen.
			\item Es Sollten keine Schwierigkeiten oder ungewöhnliche Geräusche auftreten.
			\item Bei Problemen Schrauben der Halterung für die Welle lösen, Riemen spannen und Schrauben wieder festziehen.
		\end{itemize}
		\item \textbf{Stromversorgung:}
		\begin{itemize}
			\item Netzkabel anschließen: Kupplung in Netzbuchse, Stecker in Steckdose.
		\end{itemize}
		\item \textbf{Einschalten:}
		\begin{itemize}
			\item Gerät über Power-Schalter einschalten.
			\item Display zeigt Bitte warten und wechselt zu Bewegungsstufe 1.
			\item Status-LED leuchtet grün, wenn das System bereit ist.
		\end{itemize}
		\item \textbf{Bewegungsstufe auswählen:}
		\begin{itemize}
			\item Drehschalter drehen, um Bewegungsstufe auszuwählen.
			\item Auswahl durch nicht weiteres verdrehen des Drehschalters.
		\end{itemize}
		\item \textbf{Starten:}
		\begin{itemize}
			\item Start-/Stopp-Taster betätigen, um den Ablauf zu starten.
			\item Status-LED wechselt zu gelb (Referenzfahrt) und dann zu blau (Demonstrationsablauf).
		\end{itemize}
		\item \textbf{Stoppen:}
		\begin{itemize}
			\item Ablauf stoppt automatisch oder durch erneutes Drücken des Start-/Stopp-Tasters.
			\item Status-LED wechselt zu grün und ist bereit für den nächsten Ablauf.
		\end{itemize}
	\end{enumerate}
	
		\section{Bewegungsablauf}
	
\fontsize{8}{10}\selectfont
\begin{tabularx}{\textwidth}{|p{3cm}|X|X|X|p{1cm}|X|}
	\hline 
	\textbf{Drehschalterstellung}  & \textbf{Bewegungsstufe} & \textbf{Max Geschwindigkeit [m/s]} &  \textbf{Max Beschleunigung [m/s$^{2}$]} \\ \hline
	1 & Von 0\ cm auf 35\ cm & 5 & 2,3   \\
	\hline
	2 & Von 0\ cm auf 30\ cm & 5 & 2,3 \\
	\hline
	3 & Von 0\ cm auf 25\ cm & 5 & 2,3 \\
	\hline
	4 & Von 0\ cm auf 20\ cm & 5 & 2,3 \\
	\hline
	5 & Von 0\ cm auf 15\ cm & 5 & 2,3 \\
	\hline
	6 & Pendelt Von 0\ cm auf 35\ cm & 1 & 2 \\
	\hline
	7 &Pendelt Von 0\ cm auf 35\ cm & 2 & 3 \\
	\hline
	8 & Pendelt Von 0\ cm auf 35\ cm & 3 & 4 \\
	\hline
	9 & Pendelt Von 0\ cm auf 35\ cm & 4 & 5 \\
	\hline
	10 & Pendelt Von 0\ cm auf 35\ cm & 5 & 6 \\
	\hline
	
	
\end{tabularx}
	
	


	



	

	


	
\end{document}
