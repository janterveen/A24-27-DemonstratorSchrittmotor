%%%%%%%%%%%%
%
% $Beschreibung: Beschreibung der Inbetriebnahme $
% $Autor: Theilmann und ter Veen $
% $Datum: 02.05.2024 $
% $Pfad: DemonstratorSchrittmotor/Manual/Chapter/de/SetUp.tex $
% $Version: 2 $
%
%%%%%%%%%%%%

\chapter{Inbetriebnahme}

\begin{enumerate} \label{STU}

	\item \textbf{Elastische Einstellung des Riemens}: Überprüfen Sie die Spannung des \textbf{Riemens}, indem Sie den \textbf{Schlitten} manuell bewegen. Wenn während der Bewegung Schwierigkeiten oder ungewöhnliche Geräusche auftreten, lösen Sie mithilfe eines Sechskantschlüssels die zwei \textbf{Einstellschrauben} auf der Frontseite und die zwei \textbf{Einstellschrauben} auf der Heckseite entgegen dem Uhrzeigersinn. Drücken Sie mit ihrer Handkraft den \textbf{Einstellschlitten} soweit von der Vorderblende aus gesehen nach rechts, bis der Riemen ausreichend gespannt ist. Währenddessen ziehen Sie die beiden Einstellschrauben auf der Frontseite im Uhrzeigersinn fest. Überprüfen Sie, ob ein reibungsloses Gleiten des \textbf{Schlittens} vorhanden und ob der Riemen gespannt ist. Im Anschluss ziehen Sie die \textbf{Einstellschrauben} auf der Heckseite im Uhrzeigersinn fest. 
	
	\item \textbf{Stromversorgung:} Stecken Sie dazu die Kupplung des \textbf{Netzkabels} in die Netzbuchse auf der Heckseite des Demonstrators, so dass dieser fest sitzt. Schließen Sie danach das \textbf{Netzkabel} an Ihre Stromversorgung (230\ V) an.
	
	\item \textbf{Einschalten}: Schalten Sie das Gerät über den \textbf{Power-Schalter} ein. Das \textbf{Display} zeigt zunächst den Status \glqq \textit{Start...}\grqq \ an, bis es nach einer kurzen Pufferzeit die voreingestellte \textit{\glqq Startbereit Stufe 1\grqq} anzeigt.
	Die \textbf{Status-LED} zeigt parallel dazu ebenfalls den Status des Demonstrators an. Sobald das System bereit ist, leuchtet die LED \textit{rot}.
\begin{center}	
	\begin{tikzpicture}
	% Grauer Hintergrund
	\fill[mygray] (-1,-1) rectangle (10,3);
	
	% Linkes Panel
	\fill[black] (0,0) rectangle (4,2);
	\node[align=center, white, text=myblue, scale=1.4] at (1,1.5) {Start...};
	
	% Rechtes Panel
	\fill[black] (5,0) rectangle (9,2);
	\node[align=center, white, text=myblue, scale=1.2] at (6.3,1.5) {Startbereit};
	\node[align=center, white, text=myblue, scale=1.2] at (5.9,0.5) {Stufe 1};
	
	% Weißer Strich
	\draw[white, thick] (4.5,-1) -- (4.5,3);
\end{tikzpicture}
\end{center}
	
	\item \textbf{Auswahl der Bewegungsstufe}: Wählen Sie nun über den \textbf{Drehschalter} einer der zehn verschiedenen Bewegungsstufen (nähere Informationen dazu in Kapitel \ref{bew} \nameref{bew}) aus, indem Sie den Drehschalter drehen. Wird der Drehschalter um einen Schritt im Uhrzeigersinn gedreht, zeigt das \textbf{Display} die nächsthöhere Bewegungsstufe an. Wird der Drehschalter um einen Schritt entgegen den Uhrzeigersinn gedreht, zeigt das \textbf{Display} die nächstniedrigere Bewegungsstufe an.	
	
	\item \textbf{Starten}: Durch Betätigen des \textbf{Start-Tasters} können Sie nun den Demonstrationsablauf starten. Zunächst wird eine notwendige Referenzfahrt ausgeführt. Nach der Referenzfahrt beginnt der Demonstrationsablauf.
	\begin{center}	
		\begin{tikzpicture}
			% Grauer Hintergrund
			\fill[mygray] (-1,-1) rectangle (5,2);
			
			% Linkes Panel
			\fill[black] (0,-0.5) rectangle (4,1.5);
			\node[align=center, white, text=myblue, scale=1.4] at (1.5,1) {Programm };
			\node[align=center, white, text=myblue, scale=1.4] at (2,0.5) {wird ausgeführt};
			\node[align=center, white, text=myblue, scale=1.4] at (1,0) {Stufe 1};
			
		\end{tikzpicture}
	\end{center}
	
	\item \textbf{Stoppen}: Der Demonstrationsablauf stoppt automatisch am Ende des Bewegungszyklus. Ist der Ablauf ohne Störung abgeschlossen, ist der Demonstrator bereit für den nächsten Ablauf.
\end{enumerate}


