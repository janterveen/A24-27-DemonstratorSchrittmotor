%%%%%%%%%%%%
%
% $Autor: Theilmann und ter Veen $
% $Datum: 02.05.2024 $
% $Pfad: DemonstratorSchrittmotor\Manual\Chapters\de\SetUp.tex $
% $Version: 2 $
% !TeX spellcheck = en_GB/de_DE
% !TeX encoding = utf8
% !TeX root = HandbuchDemonstratorSchrittmotor 
% !TeX TXS-program:bibliography = txs:///BibTex
%
%%%%%%%%%%%%

\chapter{Inbetriebnahme}
	\textit{(Es müssen noch Bilder für die Einstellung des Riemens hier rein und Bilder zu den einzelnen Punkten evtl. mit Ziffern zur Referenzierung, z.B. Powerschalter (1) und im Bild ist dann eine (1) beim Schalter)}\\
%möglicherweise wird noch eine Spannhilfe benötigt oder es muss noch eine 2.Person festhalten, damit der Riemen eingestellt werden kann
\begin{enumerate} \label{STU}

	\item \textbf{Elastische Einstellung des Riemens}: Überprüfen Sie die Spannung des \textbf{Riemens}, indem Sie den \textbf{Schlitten} manuell bewegen. Wenn während der Bewegung Schwierigkeiten oder ungewöhnliche Geräusche auftreten, lösen Sie mithilfe eines Sechskantschlüssels die zwei \textbf{Einstellschrauben} auf der Frontseite und die zwei \textbf{Einstellschrauben} auf der Heckseite entgegen den Uhrzeigersinn. Drücken Sie mit ihrer Handkraft den \textbf{Einstellschlitten} soweit von der Vorderblende aus gesehen nach rechts, bis der Riemen ausreichend gespannt ist. Währenddessen ziehen Sie die beiden Einstellschrauben auf der Frontseite im Uhrzeigersinn fest. Überprüfen Sie, ob ein reibungsloses Gleiten des \textbf{Schlittens} vorhanden und ob der Riemen gespannt ist. Im Anschluss ziehen Sie die \textbf{Einstellschrauben} auf der Heckseite im Uhrzeigersinn fest. 
	
	\item \textbf{Einschalten}: Schalten Sie nun das Gerät über den \textbf{Power-Schalter} ein. Das \textbf{Display} zeigt zunächst den Status \glqq \textit{Bitte warten...}\grqq \ an, bis es nach einer kurzen Pufferzeit die voreingestellte \textit{\glqq Bewegungsstufe 1\grqq} anzeigt.
	Die \textbf{Status-LED} zeigt parallel dazu ebenfalls den Status des Demonstrators an. Sobald das System bereit ist, leuchtet die LED \textit{grün}.
	
	\item \textbf{Auswahl der Bewegungsstufe}: Wählen Sie nun über den \textbf{Drehschalter} eine der zehn verschiedenen Bewegungsstufen (nähere Informationen dazu in Kapitel \ref{bew} \nameref{bew}) aus, indem Sie den Drehschalter drehen. Wird der Drehschalter um einen Schritt im Uhrzeigersinn gedreht, zeigt das \textbf{Display} die nächsthöhere Bewegungsstufe an. Wird der Drehschalter um einen Schritt entgegen den Uhrzeigersinn gedreht, zeigt das \textbf{Display} die nächstniedrigere Bewegungsstufe an. Sobald Sie die gewünschte Bewegungsstufe erreicht haben, bestätigen Sie dies durch Eindrücken des Drehschalters.	
	
	\item \textbf{Starten}: Durch Betätigen des \textbf{Start-/Stopp-Tasters} können Sie nun den Demonstrationsablauf starten. Zunächst wechselt die \textbf{Status-LED} auf \textit{gelb}, da zu Beginn eine notwendige Referenzfahrt ausgeführt wird. Sobald die \textbf{Status-LED} auf \textit{blau} wechselt, beginnt der Demonstrationsablauf.
	
	\item \textbf{Stoppen}: Der Demonstrationsablauf stoppt automatisch am Ende des Bewegungszyklus. Wenn sie jedoch den Demonstrationsablauf abbrechen möchten, so betätigen Sie erneut den Start-/Stopp-Taster. Ist der Ablauf ohne Störung abgeschlossen, dann wechselt die \textbf{Status-LED} auf \textit{grün} und ist bereit für den nächsten Ablauf.
\end{enumerate}
