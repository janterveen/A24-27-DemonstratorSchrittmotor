%%%%%%%%%%%%
%
% $Autor: Theilmann $
% $Datum: 30.04.2024 $
% $Pfad: DemonstratorSchrittmotor\Manual\Chapters\de\Disp.tex $
% $Version: 1 $
% !TeX spellcheck = en_GB/de_DE
% !TeX encoding = utf8
% !TeX root = HandbuchDemonstratorSchrittmotor 
% !TeX TXS-program:bibliography = txs:///BibTex
%
%%%%%%%%%%%%

\chapter{Display}
Im eingeschalteten Zustand zeigt das \textbf{Display} die aktuell ausgewählte Bewegungsstufen an. Wird am \textbf{Drehschalter} gedreht, ändert sich die ausgewählte Bewegungsstufe. Wird der Drehschalter im Uhrzeigersinn um einen Schritt gedreht, zeigt das \textbf{Display} die nächst höhere Bewegungsstufe. Wird der Drehschalter entgegen den Uhrzeigersinn um einen Schritt gedreht, zeigt das \textbf{Display} die nächst niedrigere Bewegungsstufe.

(Platzhalter für ein Bild)
%Es folgt ein Bild von dem OLED-Display mit einer ausgewählten Bewegungsstufe
%\begin{center}
%	\includegraphics[width=\textwidth]{Images/-----.png}
	%	\caption{Dies ist eine Konzeptskizze und wird noch ausgetauscht} \label{-}
%\end{center}

%Es folgt ein Bild von dem OLED-Display mit Running 
%\begin{center}
%	\includegraphics[width=\textwidth]{Images/-----.png}
	%	\caption{Dies ist eine Konzeptskizze und wird noch ausgetauscht} \label{-}
%\end{center}

 Wird nicht weiter gedreht, ist die auf dem \textbf{Display} angezeigte Bewegungsstufe, die ausgewählte Bewegungsstufe. Nach Beginn eines Bewegungsablaufes zeigt das \textbf{Display} \textbf{Running} an. Nach Beenden eines Bewegungsablauf zeigt das \textbf{Display} wieder die akutell ausgewählte Bewegungsstufe an. 

