%%%%%%%%%%%%
%
% $Beschreibung: Beschreibung des Bewegungablaufes und der Bewegungsstufen $
% $Autor: Theilmann $
% $Datum: 19.04.2024 $
% $Pfad: DemonstratorSchrittmotor/Manual/Chapter/de/Bewegungsablauf.tex $
% $Version: 3 $
%
%%%%%%%%%%%%


\chapter{Bewegungsablauf}

In zehn verschiedenen Bewegungsstufen (siehe Kapitel \ref{bew} \nameref{bew}) zeigt der \textbf{Demonstrator} die Möglichkeiten eines Schrittmotors. Es werden in einer Bewegungsstufe folgende Bewegungscharakteristiken demonstriert:  

\begin{itemize}
	\item \textbf{Bewegungsstufen 1-10: Zyklen fahren (Der \textbf{Schlitten} fährt mit der ausgewählten Geschwindigkeit zwischen zwei Punkten)}
	\begin{itemize}
		\item\textbf{Beschleunigung}: Der \textbf{Schlitten} beschleunigt. Folglich erhöht sich die Geschwindigkeit des \textbf{Schlittens} bis zu dem maximalen Wert. 
		\item\textbf{Konstante Geschwindigkeit}: Nach dem Erreichen der ausgewählten Geschwindigkeit, bewegt sich der \textbf{Schlitten} mit konstanter Geschwindigkeit weiter.
		\item\textbf{Verzögerung}: Der \textbf{Schlitten} verzögert. Die Geschwindigkeit des \textbf{Schlittens} nimmt ab.
		\item\textbf{Zyklen}: Der \textbf{Schlitten} wiederholt diesen Ablauf .
	\end{itemize}
\end{itemize}



