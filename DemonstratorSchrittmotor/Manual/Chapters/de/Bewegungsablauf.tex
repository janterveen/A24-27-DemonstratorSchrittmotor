% $Autor: Theilmann $
% $Datum: 2024-19-04 12:25:11Z $
% $Pfad: Bewegungsablauf.tex $
% $Version: 1 $
% $Version: 2 $
% !TeX spellcheck = en_GB/de_DE
% !TeX encoding = utf8
% !TeX root = manual 
@@ -13,24 +13,21 @@

\chapter{Bewegungsablauf}

In Zehn verschiedenen Bewegungsstufen zeigt der \textbf{Demonstrator} die Möglichkeiten eines Schrittmotors. Es werden in einer Bewegunsstufe verschiedene Bewegungscharakteristiken demonstriert. 
In Zehn verschiedenen Bewegungsstufen (siehe Kapitel \ref{bew} \nameref{bew}) zeigt der \textbf{Demonstrator} die Möglichkeiten eines Schrittmotors. Es werden in einer Bewegungsstufe verschiedene Bewegungscharakteristiken demonstriert, die folgendermaßen ablaufen:  

\begin{itemize}
\item \textbf{Stufen 1-5}
	\begin{itemize}
\item\textbf{Beschleunigung}: Der \textbf{Schlitten} beschleunigt mit der ausgewählten Beschleunigung. Die Geschwindigkeit des \textbf{Schlittens} erhöht sich. 

	\item\textbf{Konstante Geschwindigkeit}: Nach erreichen der ausgewählten Geschwindigkeit, bewegt sich der \textbf{Schlitten} mit konstanter Geschwindigkeit weiter.

		\item\textbf{Verzögerung}: Der \textbf{Schlitten} verzögert um den ausgewählten Wert. Die Geschwindigkeit des \textbf{Schlittens} nimmt ab.

			\item\textbf{Genaue Position}: Der \textbf{Anzeigepfeil} steht beim Stoppen des \textbf{Schlittens} auf die genau ausgewählte Position.
\end{itemize}
\end{itemize}
\begin{itemize}
	\item \textbf{Stufen 6-10}
		\item \textbf{Stufen 1-5}
		\begin{enumerate}
			\item\textbf{Beschleunigung}: Der \textbf{Schlitten} beschleunigt mit dem ausgewähltem Wert. Folglich erhöht sich die Geschwindigkeit des \textbf{Schlittens} bis zu dem maximalen Wert. 
			\item\textbf{Konstante Geschwindigkeit}: Nach erreichen der ausgewählten Beschleunigung, bewegt sich der \textbf{Schlitten} mit konstanter Geschwindigkeit weiter.
			\item\textbf{Verzögerung}: Der \textbf{Schlitten} verzögert um einen festgelegten Wert. Die Geschwindigkeit des \textbf{Schlittens} nimmt ab.
			\item\textbf{Genaue Positionierung}: Der \textbf{Anzeigepfeil} steht beim Stoppen des \textbf{Schlittens} auf die genau ausgewählte Position.
		\end{enumerate}
	\end{itemize}
	\begin{itemize}
	\item\textbf{Pendeln}: Der \textbf{Schlitten} pendelt mit der ausgewählten Beschleunigung und Verzögerung vom Umkehrpunkt zum Umkehrpunkt.
\end{itemize}
\end{itemize}			
		\item \textbf{Stufen 6-10}
		\begin{itemize}
			\item\textbf{Pendeln}: Der \textbf{Schlitten} pendelt mit der ausgewählten Beschleunigung und Verzögerung zwischen zwei Punkten.
		\end{itemize}
	\end{itemize}			
