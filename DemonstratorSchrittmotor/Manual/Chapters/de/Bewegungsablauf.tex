% $Autor: Theilmann $
% $Datum: 19.04.2024 $
% $Pfad: DemonstratorSchrittmotor\Manual\Chapters\de\Bewegungsablauf.tex $
% $Version: 3 $
% !TeX spellcheck = en_GB/de_DE
% !TeX encoding = utf8
% !TeX root = HandbuchDemonstratorSchrittmotor 

\chapter{Bewegungsablauf}

In zehn verschiedenen Bewegungsstufen (siehe Kapitel \ref{bew} \nameref{bew}) zeigt der \textbf{Demonstrator} die Möglichkeiten eines Schrittmotors. Es werden in einer Bewegungsstufe verschiedene Bewegungscharakteristiken demonstriert, die folgendermaßen ablaufen:  

\begin{itemize}
	\item \textbf{Bewegungsstufen 1-5}
	\begin{itemize}
		\item\textbf{Beschleunigung}: Der \textbf{Schlitten} beschleunigt mit dem ausgewähltem Wert. Folglich erhöht sich die Geschwindigkeit des \textbf{Schlittens} bis zu dem maximalen Wert. 
		\item\textbf{Konstante Geschwindigkeit}: Nach dem Erreichen der ausgewählten Beschleunigung, bewegt sich der \textbf{Schlitten} mit konstanter Geschwindigkeit weiter.
		\item\textbf{Verzögerung}: Der \textbf{Schlitten} verzögert um einen festgelegten Wert. Die Geschwindigkeit des \textbf{Schlittens} nimmt ab.
		\item\textbf{Genaue Positionierung}: Der \textbf{Anzeigepfeil} steht beim Stoppen des \textbf{Schlittens} auf die genau ausgewählte Position.
		\item fehlt noch...
	\end{itemize}
\end{itemize}
\begin{itemize}	
	\item \textbf{Stufen 6-10}
	\begin{itemize}
		\item\textbf{Zyklen fahren}: Der \textbf{Schlitten} fährt mit der ausgewählten Beschleunigung und Verzögerung zwischen zwei Punkten.
	\end{itemize}			
\item \textbf{Stoppen des Bewegungsablaufes}: Der Bewegungsablauf stoppt automatisch. Bei den \textbf{Stufen 1-5} nach Erreichen der Position, bei den \textbf{ Stufen 6-10} nach einer Ablaufzeit von 20 Sekunden.
\end{itemize}
