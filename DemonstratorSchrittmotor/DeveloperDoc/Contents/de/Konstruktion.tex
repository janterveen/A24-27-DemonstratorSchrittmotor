%%%%%%%%%%%%
%
% $Autor: Theilmann $
% $Datum: 9.06.2024 $
% $Pfad: KonstruktionsBs.tex $
% $Version: 1 $
% !TeX spellcheck = en_GB/de_DE
% !TeX encoding = utf8
% !TeX root = manual 
% !TeX TXS-program:bibliography = txs:///biber
%
%%%%%%%%%%%%

\chapter{Konstruktion}
In diesem Kapitel wird auf die Konstruktion des kompletten Demonstrator Schrittmotors eingegangen. Zunächst werden die Rahmenbedingungen erläutert und nachfolgend wird jeweils allein auf das Gehäuse und an die Anbauteile des Aluprofils eingegangen. Im Anschluss wird auf die Komplette Baugruppe eingegangen. 

\section{Rahmenbedingungen}
Das Gehäuse sowie sämtliche Anbauteile soll mittels additiver Fertigung aus Polylactic Acid (PLA) hergestellt werden. Im und am Gehäuse werden sämtliche Komponenten zum Bedienen und Steuern des Demonstrators untergebracht. Außerdem soll durch die Konstruktion eine komfortable Handhabung ermöglicht werden. Anforderung an die Konstruktion sind der feste Sitz der Komponenten, wie beispielsweise des Tiny Machine Learning Shield mit dem aufgesetzen Arduino. 

\section{Arbeiten mit SolidWorks 2023}
Zur Herstellung des Gehäuses muss ein 3D-Modell mittels Computer Aided Design (CAD) erstellt werden. Hierfür wurde die Software SolidWorks 2023 vom französischen Software-Entwicklungsunternehmen Dassault Systèmes (DS) verwendet. " SOLIDWORKS ist eine professionelle und leistungsstarke CAD-Software, die vor allem 
im Maschinenbau viel genutzt wird." \cite{Weber.2024} Für die Additve Fertigung muss das CAD-Modell als STL-Datei gespeichert werden. 

\section{Gehäusekonstruktion}
Zu Beginn wurden die Abmaße der einzelnen Komponenten ermittelt, um die Dimensionen des Gehäuses festzustellen. Das Gehäuse besteht aus sechs Einzelteilen, erkennbar in Abbildung \ref{GehK} die miteinander mit Innensechskantschrauben DIN 912 M3 gefügt werden.

\begin{figure}[H]
	\begin{center}
		\includegraphics[width=\textwidth]{Images/Konstruktion/GehaeuseK.png}
		\caption{Baugruppe Gehäuse (Eigenaufnahme)} \label{GehK}
	\end{center}
\end{figure}

Die Plattform der Baugruppe ist die Bodenplatte mit einer Länge L von 180 mm und einer Breite B von 180 mm und einer Dicke von 7 mm, erkennbar in Abbildung \ref{BodenK}. An Der Bodenplatte werden die anderen Gehäuseteile gefügt und das Schaltnetzteil, Tiny Maschine Learning Schield, Aluprofil, Schrittmotorsteuerung und der Halter für den Spannungswandler befestigt. Die Ecken sind mit einem Radius von 10 \ mm abgerundet. Die Bohrungen 1 bis 6 sind der Tabelle \ref{BohrungenGK} zu entnehmen.

\begin{figure}[H]
	\begin{center}
		\includegraphics[width=\textwidth]{Images/Konstruktion/BodenK.png}
		\caption{Bodenplatte (Eigenaufnahme)} \label{BodenK}
	\end{center}
\end{figure}

An der Bodenplatte wird die Vorderplatte gefügt. Die Vorderplatte hat eine Breite B von 170 \ mm und eine Höhe H von 65 \ mm, erkennbar in Abbildung \ref{VorneK}. Das OLED-Display wird von innen eingesetzt und verschraubt mit DIN 912 M3$\times$6 Schrauben (s. Markierung a $25 \times 15 \ mm$  in Abb.\ref{VorneK}). Die LED wird von innen eingesetzt und mittels Presspassung fixiert (s. Markierung b \O $ \ 8 \ mm$ in Abb.\ref{VorneK}). Der Drehwinkel-Encoder wird von innen durchgeführt und mittels einer P4-Mutter befestigt (s. Markierung c \O $ \ 7 \ mm$  in Abb.\ref{VorneK}). Der Drucktaster wird von außen durchgeführt und mittels Kontermutter befestigt (s. Markierung a \O $ \ 13,5 \ mm$  in Abb.\ref{VorneK}). Die Bohrungen 7 bis 9 sind der Tabelle \ref{BohrungenGK} zu entnehmen.  


\begin{figure}[H]
	\begin{center}
		\includegraphics[width=\textwidth]{Images/Konstruktion/VorneK.png}
		\caption{Vorderplatte (Eigenaufnahme)} \label{VorneK}
	\end{center}
\end{figure}

An der Vorderenplatte und Hinterplatte wird die Seitenplatte Links gefügt. Die Seitenplatte Links hat eine Länge von 177,32 \ mm, eine Höhe von 65 \ mm und eine Dicke von 5 \ mm, erkennbar in Abbildung \ref{SeiteLK}. Die Ecken sind mit einem Radius von 10 \ mm abgerundet. Die Bohrungen 10 und 11 sind der Tabelle \ref{BohrungenGK} zu entnehmen. 


\begin{figure}[H]
	\begin{center}
		\includegraphics[width=\textwidth]{Images/Konstruktion/SeiteLK.png}
		\caption{Seitenplatte Links (Eigenaufnahme)} \label{SeiteLK}
	\end{center}
\end{figure}

Die Hinterplatte wird an der Bodenplatte gefügt. Die Hinterplatte hat eine Breite B von 170 \ mm und eine Höhe H von 65 \ mm, erkennbar in Abbildung \ref{HinterK}. Der Kaltgeräteanschluss mit Wippschalter von Artillery wird von außen der Hinterplatte montiert, (s. Markierung e $30,5 \times 47 \ mm$  in Abb.\ref{HinterK}). Die Bohrungen 12 bis 15 sind der Tabelle \ref{BohrungenGK} zu entnehmen.

\begin{figure}[H]
	\begin{center}
		\includegraphics[width=\textwidth]{Images/Konstruktion/HinterK.png}
		\caption{Hinterplatte (Eigenaufnahme)} \label{HinterK}
	\end{center}
\end{figure}

An der Vorderenplatte und Hinterplatte wird die Seitenplatte Rechts gefügt. Die Seitenplatte Rechts hat eine Länge von 177,32 \ mm, eine Höhe von 65 \ mm und eine Dicke von 5 \ mm, erkennbar in Abbildung \ref{SeiteLK}. Die Ecken sind mit einem Radius von 10 \ mm abgerundet. Für die Durchführung der Leitungen vom Schrittmotor und des Microschalters wurde eine Aussparung konstruiert (s. Markierung f \O $ \ 13 \ mm$  in Abb.\ref{SeiteRK}). Die Bohrungen 16 und 17 sind der Tabelle \ref{BohrungenGK} zu entnehmen. 


\begin{figure}[H]
	\begin{center}
		\includegraphics[width=\textwidth]{Images/Konstruktion/SeiteRK.png}
		\caption{Seitenplatte Rechts (Eigenaufnahme)} \label{SeiteRK}
	\end{center}
\end{figure}

Die Deckelplatte wird an der Vorder- und Hinterplatte gefügt. Die Deckelplatte hat ein Länge L von 180 mm und einer Breite B von 180 \ mm und einer Dicke von 7 \ mm, erkennbar in Abbildung \ref{DeckelK}. Die Ecken sind mit einem Radius von 10 \ mm abgerundet. Die Bohrungen 18 sind der Tabelle \ref{BohrungenGK} zu entnehmen.

\begin{figure}[H]
	\begin{center}
		\includegraphics[width=\textwidth]{Images/Konstruktion/DeckelK.png}
		\caption{Deckelplatte (Eigenaufnahme)} \label{DeckelK}
	\end{center}
\end{figure}


\begin{figure}[H]
\begin{center}
	\fontsize{8}{10}\selectfont
	\begin{tabularx}{\textwidth}{|p{0.4cm}|p{1.2cm}|X|X|X|X|} 
		\hline 
		\textbf{Nr.} & \textbf{\O} & \textbf{Beschreibung} \\ \hline
		1 & DIN912 M3 & Befestigung Schaltnetzteil an Bodenplatte \\ \hline
		2 & DIN912 M3 & Befestigung Tiny Maschine Learning Schield an Bodenplatte \\ \hline
		3 & DIN912 M3 & Fügen der Gehäuseteile an Bodenplatte  \\ \hline
		4 & DIN912 M3 & Befestigung Bodenplatte an Aluprofil \\ \hline
		5 & DIN912 M3 & Befestigung Schrittmotorsteuerung an Bodenplatte \\ \hline
		6 & DIN912 M3 & Befestigung des Halters für Spannungswandler an Bodenplatte \\ \hline
		7 & 4 \ mm & Bohrung für M3-Einpressmutter zur Befestigung Vorderplatte an Bodenplatte \\ \hline
		8 & 4 \ mm & Bohrung für M3-Einpressmutter zur Befestigung Vorderplatte an Deckelplatte \\ \hline
		9 & 4 \ mm & Bohrung für M3-Einpressmutter zur Befestigung Vorderplatte an Seitenplatte Links \\ \hline
		10& DIN912 M3 & Befestigung der Seitenplatte Links an Hinterplatte \\ \hline
		11& DIN912 M3 & Befestigung der Seitenplatte Links an Vorderplatte \\ \hline
		12& 4 \ mm & Bohrung für M3-Einpressmutter zur Befestigung Hinterplatte an Deckelplatte \\ \hline
		13& 4 \ mm & Bohrung für M3-Einpressmutter zur Befestigung Hinterplatte an Bodenplatte \\ \hline
		14& 4 \ mm & Bohrung für M3-Einpressmutter zur Befestigung Hinterplatte an Seitenplatte Links \\ \hline
		15& 4 \ mm & Bohrung für M3-Einpressmutter zur Befestigung Hinterplatte an Seitenplatte Rechts \\ \hline
		16& DIN912 M3 & Befestigung der Seitenplatte Rechts an Hinterplatte \\ \hline
		17& DIN912 M3 & Befestigung der Seitenplatte Rechts an Vorderplatte \\ \hline
		18& DIN912 M3 & Fügen der Gehäuseteile an Deckelplatte \\ \hline
	\end{tabularx}
	\captionof{table}{Bohrungen im Gehäuse}	\label{BohrungenGK}
\end{center}
\end{figure}

\section{Anbauteile}

Nachfolgend werden die Anbauteile in der Konstruktion erläutert. 
In der Abbildung \ref{FlankeMotorK} zu erkennen, ist die Halterung für den Motor. Die Halterung hat ein Breite B von 50 \ mm und eine Höhe H von 70 mm und eine Dicke von 5 \ mm, erkennbar in Abbildung \ref{FlankeMotorK}. Außerdem hat die Halterung eine Aussparung für den Motor (s. Markierung g Motor\O $ \ 22,5 \ mm$, Welle \O $ \ 6 \ mm$  in Abb.\ref{SeiteRK}). Die Halterung wird an dem Aluprofil montiert. Die Bohrungen 20 und 21 sind der Tabelle \ref{BohrungenAK} zu entnehmen. 
 
\begin{figure}[H]
	\begin{center}
		\includegraphics[width=\textwidth]{Images/Konstruktion/FlankeMotorK.png}
		\caption{Halterung für den Motor (Eigenaufnahme)} \label{FlankeMotorK}
	\end{center}
\end{figure}
 
Die Flanke für die Welle hat eine Breite B von 50 \ mm und eine Höhe H von \ 70 mm und eine Dicke von 5 \ mm , erkennbar in Abbildung \ref{FlankeK}. Außerdem hat die Halterung eine Aussparung für die Welle (s. Markierung h \O $ \ 5 \ mm$ in Abb.\ref{FlankeK}). Die Halterung wird an dem Aluprofil montiert. Die Bohrungen 22 sind der Tabelle \ref{BohrungenAK} zu entnehmen.  
 
\begin{figure}[H]
	\begin{center}
		\includegraphics[width=\textwidth]{Images/Konstruktion/FlankeK.png}
		\caption{Halterung für die Welle (Eigenaufnahme)} \label{FlankeK}
	\end{center}
\end{figure} 
 
Der Griff zum Transportieren hat eine Breite B von 170 \ mm und eine Höhe H von 60 \ mm bei einer Dicke von 20 \ mm, die innere Spannweite beträgt Li 100 \ mm mit einer Höhe Hi von  50 \ mm , erkennbar in Abbildung \ref{GriffK}. Die Bohrungen 23 sind der Tabelle \ref{BohrungenAK} zu entnehmen.  

\begin{figure}[H]
	\begin{center}
		\includegraphics[width=\textwidth]{Images/Konstruktion/GriffK.png}
		\caption{Transportgriff (Eigenaufnahme)} \label{GriffK}
	\end{center}
\end{figure}  

Der Drehknopf wird mittels Presspassung und DIN912 M3 Schraube auf dem Drehwinkel-Encoder befestigt. Der Drehknopf hat einen Durchmesser D von 32 \ mm bei einer Länge von 20 \ mm und ist an zeigt eine Zahnförmige Kontur, erkennbar in Abbildung \ref{DrehKnopfK}. Die Bohrungen 24 sind der Tabelle \ref{BohrungenAK} zu entnehmen.

\begin{figure}[H]
	\begin{center}
		\includegraphics[width=\textwidth]{Images/Konstruktion/DrehKnopfK.png}
		\caption{Drehknopf (Eigenaufnahme)} \label{DrehKnopfK}
	\end{center}
\end{figure}  

Der Anzeiger hat einer Breite B von 40 \ mm bei einer Länge L von 50 \ mm, erkennbar in Abbildung \ref{AnzeigerK}. Der Pfeil (s. Markierung i in Abb.\ref{AnzeigerK}) auf dem Anzeiger hat eine Tiefe von 35 \ mm. Die Bohrungen 25 und 26 sind der Tabelle \ref{BohrungenAK} zu entnehmen.

\begin{figure}[H]
	\begin{center}
		\includegraphics[width=\textwidth]{Images/Konstruktion/AnzeigerK.png}
		\caption{Anzeiger (Eigenaufnahme)} \label{AnzeigerK}
	\end{center}
\end{figure}  
 
 \begin{figure}[H]
 	\begin{center}
 		\fontsize{8}{10}\selectfont
 		\begin{tabularx}{\textwidth}{|p{0.4cm}|p{1.2cm}|X|X|X|X|} 
 			\hline 
 			\textbf{Nr.} & \textbf{\O} & \textbf{Beschreibung} \\ \hline
 			20 & DIN912 M3 & Befestigung Motor an Flanke Halterung für Motor \\ \hline
 			21 & DIN912 M3 & Befestigung Flanke Halterung für Motor an Aluprofil \\ \hline
 			22 & DIN912 M3 & Befestigung Flanke an Aluprofil  \\ \hline
 			23 & DIN912 M3 & Befestigung Griff an Aluprofil \\ \hline
 			24 & DIN912 M3 & Befestigung Drehknopf mit Drehwinkel-Encoder \\ \hline
 			25 & DIN912 M3 & Befestigung Anzeiger an Schlitten \\ \hline
 			26 & 2,8 \ mm & Bohrung für M3-Einpressmutter zur Befestigung Riemen an Anzeiger \\ \hline
 		\end{tabularx}
 		\captionof{table}{Bohrungen der Anbauteile}	\label{BohrungenAK}
 	\end{center}
 \end{figure}


\section{3D-Druck mit PLA}
Die Einzelteile ds Gehäuses sowie die Anbauteile wurde mit dem Anycubic Kobra 2 Neo gefertigt, wahrnehmbar in \ref{AKK2N}. Als Fertigungsmaterial wurde PLA gewählt. Das additve Verfahren eignet sich als kostengüstiges und schnelles Verfahren für den Prototypen-Bau. Für den Druck aller Einzelteile wurde c.a 1kg Filament verbraucht, dabei liegt der Preis für einen kg bei ungefähr 20 €. Der Längste Druck dauerte ca. 2,5h. 

 \begin{figure}[H]
 	\begin{center}
 		\includegraphics[width=\textwidth]{Images/Konstruktion/AKK2N.jpg}
 		\caption{AnyKubic 2 Kobra Neo (Eigenaufnahme)} \label{AKK2N}
 	\end{center}
 \end{figure}  