%%%%%%%%%%%%%%%%%%%%
%
% $Beschreibung: Erklärung und Durchführung der Tests $
% $Autor: Theilmann und Hanneken $
% $Datum: 15.05.2024 $
% $Pfad: DemonstratorSchrittmotor/DeveloperDoc/Contents/de/Testdurch.tex $
% $Version: 1 $
%
%
%%%%%%%%%%%%%%%%%%%
\chapter{Testdurchläufe}

Ein fehlerfreier Ablauf kann nie vollständig gewährleistet werden, es können stets zufällige Fehler auftreten. Um die Funktionalität und Zuverlässigkeit des Systems zu gewährleisten, sind regelmäßige Testdurchläufe notwendig. Eine erste Überprüfung findet mit der Sichtprüfung der Hardware auf Schäden statt. Für die softwareseitige Überprüfung können Testprogramme nach Einschalten durchlaufen werden.  


\section{Arduino Nano 33 BLE Sense Lite}

Für die Überprüfung des Arduinos, werden Beispielsketche von der Arduino IDE zur Verfügung gestellt. Unter dem Pfad File/Examples/01.Basics können Beispielsketche ausgesucht werden. Mit dem Beispielprogramme Blink wird die Built In LED angesprochen. Bei richtiger Funktion des Arduinos sollte die LED für eine Sekunde eingeschaltet und für eine Sekunde ausgeschaltet werden. Dieses wiederholt sich fortlaufend. Ist die Ansteuerung des Arduinos sichergestellt. 

% Code einfügen
%
%

\section{Schaltnetzteil}

Die Überprüfung des Schaltnetzteils \textbf{muss} von einer Elektrofachkraft durchgeführt werden. Ist das Schaltnetzteil ordnungsgemäß angeschlossen, leuchtet eine Signalleuchte Grün auf dem Schaltnetzteil. Um eine einwandfreie Nutzung zu gewährleisten müssen noch die Ausgangsspannungen des Netzteils mit einem Voltmeter überprüft werden. Liegen die Ausgangsspannungen jeweils bei 12 \ V und 5 V \ ist eine einwandfreie Nutzung gewährleistet. 


\section{Spannungsregler AMS1117}

Ist der Spannungswandler ordnungsgemäß angeschlossen, leuchtet eine Signalleuchte Rot auf dem AMS1177. Um eine einwandfreie Nutzung zu gewährleisten muss noch die Ausgangsspannung des Spannungswandlers mit einem Voltmeter überprüft werden. Liegt die Ausgangsspannungen bei 3,3 \ V ist eine einwandfreie Nutzung gewährleistet. 


\section{ Nema 17 Schrittmotor mit Schrittmotorsteuerung A4988 }

Text

% Code einfügen
%
%

\section{OLED-Display}

Text

% Code einfügen
%
%

\section{SMD-LED}

Text

% Code einfügen
%
%

\section{Drucktaster}

Text

% Code einfügen
%
%

\section{Microschalter}

Text

% Code einfügen
%
%

\section{Drehwinkel-Encoder}

Text

% Code einfügen
%
%

\section{Demonstrator}

Text

% Code einfügen
%
%
