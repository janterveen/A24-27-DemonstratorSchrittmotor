%%%%%%%%%%%%%%%%%%%%
%
% $Beschreibung: Erklärung und Durchführung der Tests $
% $Autor: Theilmann und Hanneken $
% $Datum: 15.05.2024 $
% $Pfad: DemonstratorSchrittmotor/DeveloperDoc/Contents/de/Testdurch.tex $
% $Version: 2 $
%
%
%%%%%%%%%%%%%%%%%%%
\chapter{Testdurchläufe}

Ein fehlerfreier Ablauf kann nie vollständig gewährleistet werden, es können stets zufällige Fehler auftreten. Um die Funktionalität und Zuverlässigkeit des Systems zu gewährleisten, sind regelmäßige Testdurchläufe notwendig. Eine erste Überprüfung findet mit der Sichtprüfung der Hardware auf Schäden statt. Für die softwareseitige Überprüfung können Testprogramme nach Einschalten durchlaufen werden.  


\section{Arduino Nano 33 BLE Sense Lite}

Für die Überprüfung des Arduinos, werden Beispielsketche von der Arduino IDE zur Verfügung gestellt. Unter dem Pfad File/Examples/01.Basics können Beispielsketche ausgesucht werden. Mit dem Beispielprogramme Blink wird die Built In LED angesprochen. Bei richtiger Funktion des Arduinos sollte die LED für eine Sekunde eingeschaltet und für eine Sekunde ausgeschaltet werden. Dieses wiederholt sich fortlaufend. Ist die Ansteuerung des Arduinos sichergestellt. 

% Code einfügen
%
%

\section{Schaltnetzteil}

Die Überprüfung des Schaltnetzteils \textbf{muss} von einer Elektrofachkraft durchgeführt werden. Ist das Schaltnetzteil ordnungsgemäß angeschlossen, leuchtet eine Signalleuchte Grün auf dem Schaltnetzteil. Um eine einwandfreie Nutzung zu gewährleisten müssen noch die Ausgangsspannungen des Netzteils mit einem Voltmeter überprüft werden. Liegen die Ausgangsspannungen jeweils bei 12 \ V und 5 V \ ist eine einwandfreie Nutzung gewährleistet. 


\section{Spannungsregler AMS1117}

Ist der Spannungswandler ordnungsgemäß angeschlossen, leuchtet eine Signalleuchte Rot auf dem AMS1177. Um eine einwandfreie Nutzung zu gewährleisten muss noch die Ausgangsspannung des Spannungswandlers mit einem Voltmeter überprüft werden. Liegt die Ausgangsspannungen bei 3,3 \ V ist eine einwandfreie Nutzung gewährleistet. 


\section{ Nema 17 Schrittmotor mit Schrittmotorsteuerung A4988 }

Text

% Code einfügen
%
%

\section{OLED-Display}

Text

% Code einfügen
%
%

\section{SMD-LED}

Text

% Code einfügen
%
%

\section{Drucktaster}

Text

% Code einfügen
%
%

\section{Microschalter}

Text

% Code einfügen
%
%

\section{Drehwinkel-Encoder}

Text

% Code einfügen
%
%



\section{Mittlere Geschwindigkeit des Schlittens}

Um die mittlere Geschwindigkeit des Schlittens herauszufinden, muss die Strecke durch die Zeit dividiert werden. Für die Tests wurde der Demonstrator aufgebaut und die Stufen jeweils fünf mal gefahren. In der Stufe wurde die Zeit, die der Schlitten vom Wendepunkt bis zum Wendepunkt benötigt gestoppt. Die Strecke beträgt 0,35\ m. Gestoppt wurde mit einem Handelsüblichen Stoppuhr. Die Ergebnisse der Tests sind in der Tabelle \ref{TestZeitTab} aufgeführt.

\begin{figure}[H]
\begin{center}
	\fontsize{8}{10}\selectfont
	\begin{tabularx}{\linewidth}{|p{0.8cm}|X|X|X|X|X|X|}
		\hline 
		\textbf{Stufe} & \textbf{Ergebnis Versuch 1 [s]} & \textbf{Ergebnis Versuch 2 [s]} & \textbf{Ergebnis Versuch 3 [s]}& \textbf{Ergebnis Versuch 4 [s]} & \textbf{Ergebnis Versuch 5 [s]} \\ \hline
		
		1 & 3,54 & 3,61 & 3,62 & 3,63 & 3,58 \\ \hline
		2 & 2,43 & 2,35 & 2,42 & 3,37 & 2,32 \\ \hline
		3 & 1,95 & 1,83 & 1,86 & 1,97 & 1,89 \\ \hline
		4 & 1,43 & 1,58 & 1,62 & 1,64 & 1,89 \\ \hline
		5 & 1,39 & 1,38 & 1,48 & 1,51 & 1,31 \\ \hline
		6 & 1,27 & 1,24 & 1,23 & 1,25 & 1,21 \\ \hline
		7 & 1,16 & 1,12 & 1,17 & 1,11 & 1,13 \\ \hline
		8 & 0,97 & 1,05 & 1,08 & 1,05 & 1,09 \\ \hline
		9 & 1,06 & 0,96 & 1,01 & 1,01 & 1,03 \\ \hline
		10 & 0,96 & 0,96 & 0,94 & 1,00 & 0,98 \\ \hline
		
		\end{tabularx}
			\captionof{table}{Ergebnisse Messung der Zeit eines Weges der einzelnen Stufen}
			\label{TestZeitTab}
		\end{center}
\end{figure}

Mit den Ergebnisse aus der Tabelle \ref{TestZeitTab} kann die mittlere Geschwindigkeit errechnet werden. Die Strecke dividiert mit der Zeit ergibt die mittlere Geschwindigkeit, erkenntlich in Tabelle \ref{TestMeterSekunde}. Dabei ist zu beachten dass die Ergebnisse in Meter pro Sekunde dargestellt sind. 

\begin{figure}[H]
	\begin{center}
		\fontsize{8}{10}\selectfont
		\begin{tabularx}{\linewidth}{|p{0.8cm}|X|X|X|X|X|X|}
			\hline 
			\textbf{Stufe} & \textbf{Ergebnis Versuch 1 [m/s]} & \textbf{Ergebnis Versuch 2 [m/s]} & \textbf{Ergebnis Versuch 3 [m/s]}& \textbf{Ergebnis Versuch 4 [m/s]} & \textbf{Ergebnis Versuch 5 [m/s]} & \textbf{\O [m/s]}\\ \hline
			
			1 & 0,0988 & 0,0969 & 0,0966 & 0,0964 & 0,0977 & 0,0973 \\ \hline
			2 & 0,1440 & 0,1489 & 0,1446 & 0,1476 & 0,1508 & 0,1472 \\ \hline
			3 & 0,1794 & 0,1912 & 0,1881 & 0,1776 & 0,1851 & 0,1843 \\ \hline
			4 & 0,2447 & 0,2215 & 0,2160 & 0,2134 & 0,2201 & 0,2231 \\ \hline
			5 & 0,2517 & 0,2536 & 0,2364 & 0,2317 & 0,2671 & 0,2481\\ \hline
			6 & 0,2755 & 0,2822 & 0,2845 & 0,2800 & 0,2892 & 0,2823 \\ \hline
			7 & 0,3017 & 0,3125 & 0,2991 & 0,3465 & 0,3097 & 0,3076 \\ \hline
			8 & 0,3608 & 0,3333 & 0,3240 & 0,3333 & 0,3211 & 0,3211 \\ \hline
			9 & 0,3301 & 0,3645 & 0,3465 & 0,3465 & 0,3398 & 0,3455 \\ \hline
			10 & 0,3645 & 0,3645 & 0,3723 & 0,3500 & 0,3571  & 0,3617 \\ \hline
			
		\end{tabularx}		
		\captionof{table}{Mittlere Geschwindigkeit und Durchschnittswert in SI-Einheiten}
		\label{TestMeterSekunde}
	\end{center}
\end{figure}

Da die Darstellung für in Meter pro Sekunde für Schrittmotoren unüblich sind, wurden diese mit in Millimeter pro Sekunde umgerechnet, erkenntlich in Tabelle \ref{TestMillimeterSekunde}. 

\begin{figure}[H]
	\begin{center}
		\fontsize{8}{10}\selectfont
		\begin{tabularx}{\linewidth}{|p{0.8cm}|X|X|X|X|X|X|}
			\hline 
			\textbf{Stufe} & \textbf{\O [m/s]} & \textbf{\O [mm/s]} \\ \hline
			
			1 & 0,0973 & 97,33  \\ \hline
			2 & 0,1472 & 147,2  \\ \hline
			3 & 0,1843 & 184,3  \\ \hline
			4 & 0,2231 & 223,1  \\ \hline
			5 & 0,2481 & 248,1  \\ \hline
			6 & 0,2823 & 282,3  \\ \hline
			7 & 0,3076 & 307,6  \\ \hline
			8 & 0,3345 & 334,5  \\ \hline
			9 & 0,3455 & 345,5  \\ \hline
			10 & 0,3617 & 361,7 \\ \hline
			
		\end{tabularx}		
		\captionof{table}{Durchschnittswerte umgerechnet in mm/s}
		\label{TestMillimeterSekunde}
	\end{center}
\end{figure}
