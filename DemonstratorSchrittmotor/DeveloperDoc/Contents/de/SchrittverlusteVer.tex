%%%%%%%%%%%%
%
% $Autor: Theilmann $
% $Datum: 2024-13-05 11:03:00Z $
% $Pfad: SchrittverlusteVer.tex $
% $Version: 1 $
% !TeX spellcheck = en_GB/de_DE
% !TeX encoding = utf8
% !TeX root = manual 
% !TeX TXS-program:bibliography = txs:///biber

\chapter{Schrittverluste Verhindern}
Um mögliche Fehlerquellen im Bezug zu Schrittverlusten auszugrenzen und zu verhindern, wird in diesem Kapitel beschrieben wie die Ursachen für Schrittverluste oder Stillstand methodisch zu ermitteln sind. 

\section{Auswahl des Schrittmotors}
Zunächst sollte ein Motor für die Anwendung gefunden werden, dabei sollten grundlegende Theoretische Regeln befolgt werden:
\begin{itemize}
	\item {\textbf{Motorauswahl druch Höchstwerte für Drehmoment und Drehzahl (Worst-Case-Szenario)}}
	\item {\textbf{Verwendung eines Sicherheitsaufschlag von 30 \% auf die Drehmoment-Drehzahl-Kennlinie(Kippmoment)}}
	\item {\textbf{Sicherstellen, dass externe Ergeinisse die Anwendung nicht blockieren können}}
\end{itemize}

Sind die geforderten Drehmomente bei den jeweiligen Drehzahlen den Motorspezifikation entsprechend, sind keine Probleme zu erwarten. Ist der Motor zu schwach gewählt und die Andwendung fordert höhere Werte, bleibt der Motor stehen. 
Der nächste Schritt ist die praktische überprüfung durch Tests. Es soll im Betrieb überprüft werden, ob Schrittverluse auftreten. Schrittmotoren verlieren Konstruktionsbedingt nicht nur einen einzigen Schritt, bei geringen Drehzahlen verliert der Motor ein Vielfaches von vier Schritten.


\section{Betriebsart}

In diesem Abschnitt werden je nach Betriebsart mögliche Ursachen erläutert, falls der Schrittmotor bei den praktischen Tests versagt.

\subsection{Start-Stopp-Betrieb}

Der Motor ist mit der Last fest verbunden und wird mit konstanter Drezahl betrieben. Innerhalb des ersten Schritts muss der Motor auf die vorgegebene Frequenz beschleunigen. 


%Bild per Tikz 
\begin{figure}[!ht]
\centering
\resizebox{1\textwidth}{!}{%
\begin{circuitikz}
\tikzstyle{every node}=[font=\small]
\draw [line width=0.5pt, ->, >=Stealth] (3.25,5.5) -- (3.25,12)node[pos=1,left, fill=white]{Frequenz};
\draw [->, >=Stealth] (3.25,5.5) -- (14.25,5.5)node[pos=1,below, fill=white]{Zeit};
\draw [ color={rgb,255:red,0; green,199; blue,252}, short] (6.5,5.5) -- (6.5,7.5);
\draw [ color={rgb,255:red,0; green,199; blue,252}, short] (6.5,7.5) -- (11.25,7.5);
\draw [ color={rgb,255:red,0; green,199; blue,252}, short] (11.25,7.5) -- (11.25,5.5);
\draw [dashed] (6.5,7.5) -- (3.25,7.5);
\node [font=\large] at (1.75,7.75) {f};
\node [font=\small] at (2.5,7.5) {Start/Stopp};
\end{circuitikz}
}%

\label{fig:Start-Stopp Frequenz}
\end{figure}
{\textbf{Fehlerbild: Motor läuft nicht an} 
\begin{center}
	\fontsize{8}{10}\selectfont
	\begin{tabularx}{\textwidth}{|p{0.4cm}|p{0.4cm}|X|X|p{1cm}|X|}
		\hline 
		\textbf{Ursachen} & \textbf{Lösungen} \\ \hline
  		Last zu Hoch & Falscher Motor, größeren Motor wählen \\
		\hline
  		Frequenz zu hoch & Reduzieren \\
		\hline
		Pendelt der Motor von Links nach Rechts, könnte eine Phase unterbrochen oder nicht angeschlossen sein & Reparieren \\
		\hline
		Phasenstrom passt nicht & Phasenstrom erhöhen, zumindest für die ersten Schritte \\
		\hline
\end{tabularx}


\subsection{Beschleunigung und Rampenprofil (Trapezförmig)}

Der Motor kann mit einer im Controller vorgegebenen Beschleunigungsrate auf bis aus die Maximalfrequenz beschleunigen. 
{\textbf{Fehlerbild: Motor läuft nicht an} (siehe Ursachen und Lösungen aus ....)% noch eine verlinkung machen ! 

{\textbf{Fehlerbild: Motor beendet die Beschleunigungsrampe nicht}
\begin{center}
	\fontsize{8}{10}\selectfont
	\begin{tabularx}{\textwidth}{|p{0.4cm}|p{0.4cm}|X|X|p{1cm}|X|}
		\hline 
		\textbf{Ursachen} & \textbf{Lösungen} \\ \hline
  		Motor bleibt bei der Resonanzfrequenz hängen & Beschleunigung erhöhen, um die Reosnansfrequenz schneller zu durchlaufen. Start-Stopp frequenz über dem Resonanzpunkt wählen. Halbschritt- oder Mikroschrittbetrieb verwenden. Mechanische Dämpfung vorsehen. \\
		\hline
  		Falsche Einstellung von Versorgungsspannung oder Strom zu gering & Spannung oder Strom erhöhen. Motor mit geringer Impendanz testen. Stromregelung verwenden. \\
		\hline
		Maximaldrehzal zu hoch & Maximaldrehzahl reduzieren. Beschleunigungsrampe abflachen \\
		\hline
		Schlechte Vorgabe der Beschleunigungsrampe durch Elektronik & Anderen Controller ausprobieren \\
		\hline
\end{tabularx}

