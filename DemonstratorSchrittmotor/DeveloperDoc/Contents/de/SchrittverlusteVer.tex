%%%%%%%%%%%%
%
% $Autor: Theilmann $
% $Datum: 13.05.2024 $
% $Pfad: SchrittverlusteVer.tex $
% $Version: 1 $
% !TeX spellcheck = en_GB/de_DE
% !TeX encoding = utf8
% !TeX root = manual 
% !TeX TXS-program:bibliography = txs:///biber
%
%%%%%%%%%%%%
\chapter{Schrittverluste Verhindern}
Um mögliche Schrittverlusten einzugrenzen und oder sie zu verhindern, wird in diesem Kapitel beschrieben, wie die Ursachen für Schrittverluste oder Stillstand methodisch zu ermitteln sind. \cite{FaulhaberDriveSystems.2020}

\section{Auswahl des Schrittmotors}
Zunächst muss ein passender Motor für die Anwendung gewählt werden. Dabei sollten folgende grundlegende Regeln erfüllt sein:
\begin{itemize}
	\item Motorauswahl durch Höchstwerte für Drehmoment und Drehzahl (Worst-Case-Szenario)
	\item Verwendung eines Sicherheitsaufschlag von 30 \% auf die Drehmoment-Drehzahl-Kennlinie(Kippmoment)
	\item Sicherstellen, dass externe Ereignisse die Anwendung nicht blockieren können
\end{itemize}

Sind die geforderten Drehmomente bei den jeweiligen Drehzahlen, den Motorspezifikation entsprechend, dann sind keine Probleme zu erwarten. Ist der Motor zu schwach gewählt und die Anwendung fordert mehr Leistung als der Motor abgeben kann, so bleibt der Motor stehen. Der nächste Schritt ist die Durchführung von Testdurchläufen. Es soll im Betrieb überprüft werden, ob Schrittverluste auftreten. Schrittmotoren verlieren konstruktionsbedingt nicht nur einen einzigen Schritt. Bei geringen Drehzahlen verliert der Motor ein Vielfaches von vier Schritten.\cite{FaulhaberDriveSystems.2020}
%% Sind es immer vielfache von vier oder wo kommt diese Zahl her?

\section{Betriebsart}

In diesem Abschnitt werden je nach Betriebsart mögliche Ursachen erläutert, falls der Schrittmotor bei den Tests versagt.

\subsection{Start-Stopp-Betrieb}

Der Motor ist mit der Last fest verbunden und wird mit konstanter Drehzahl betrieben. Innerhalb des ersten Schrittes muss der Motor auf die vorgegebene Frequenz beschleunigen.\cite{FaulhaberDriveSystems.2020} 
%% Ich würde nicht Frequenz und Drehzahl verwenden. Vorschlag: Ersetzte Frequenz überall mit Drehzahl, Drehzahl ist bei Motoren mit Kreisförmiger Bewegung sowieso üblicher
\begin{figure}[!ht]
	\centering
	\resizebox{1\textwidth}{!}{%
		\begin{circuitikz}
			\tikzstyle{every node}=[font=\LARGE]
			\draw [line width=1pt, ->, >=Stealth] (11.25,8.25) -- (11.25,22);
			\draw [ color={rgb,255:red,0; green,199; blue,252}, line width=1pt, short] (36.25,12) -- (36.25,8.25);
			\node [font=\LARGE] at (38.75,7.5) {Zeit};
			\node [font=\huge] at (8.75,12.25) {f};
			\node [font=\LARGE] at (9.75,22) {Frequenz};
			\node [font=\normalsize] at (10,12) {Start/Stopp};
			\draw [ color={rgb,255:red,0; green,199; blue,252}, line width=1pt, short] (17.5,12) -- (17.5,8.25);
			\draw [line width=1pt, ->, >=Stealth] (11.25,8.25) -- (38.75,8.25);
			\draw [ color={rgb,255:red,0; green,199; blue,252}, line width=1pt, short] (17.5,12) -- (36.25,12);
			\draw [line width=1pt, dashed] (11.25,12) -- (17.5,12);
		\end{circuitikz}
	}%

\caption{Start-Stopp Frequenz \cite{FaulhaberDriveSystems.2020}} \label{Start-Stopp Frequenz}
\end{figure}

{\textbf{Fehlerbild: Motor läuft nicht an} 
\begin{center}
	\fontsize{8}{10}\selectfont
	\begin{tabularx}{\textwidth}{|X|X|X|X|X|X|}
		\hline 
		\textbf{Ursachen} & \textbf{Lösungen} \\ \hline
		Last zu Hoch & Falscher Motor, größeren Motor wählen \\ \hline
		Frequenz zu hoch & Frequenz reduzieren\\ \hline
		Pendelt der Motor von Links nach Rechts & Es könnte eine Phase unterbrochen oder nicht angeschlossen sein, dies muss repariert werden \\ \hline
		Phasenstrom passt nicht & Phasenstrom erhöhen\\ \hline

	\end{tabularx}
		\captionof{table}{Ursachen und Lösungen: Motor läuft nicht an \cite{FaulhaberDriveSystems.2020}} \label{MOTLNA} 
\end{center}


\subsection{Beschleunigung und Rampenprofil (Trapezförmig)}

Der Motor kann mit einer im Controller vorgegebenen Beschleunigungsrate bis auf die Maximalfrequenz beschleunigen.\cite{FaulhaberDriveSystems.2020} 

\begin{figure}[!ht]
	\centering
	\resizebox{1\textwidth}{!}{%
		\begin{circuitikz}
			\tikzstyle{every node}=[font=\LARGE]
			\draw [line width=1pt, ->, >=Stealth] (11.25,8.25) -- (11.25,22);
			\draw [ color={rgb,255:red,0; green,199; blue,252}, line width=1pt, short] (17.5,12) -- (22.5,19.5);
			\draw [ color={rgb,255:red,0; green,199; blue,252}, line width=1pt, short] (22.5,19.5) -- (27.5,19.5);
			\draw [ color={rgb,255:red,0; green,199; blue,252}, line width=1pt, short] (27.5,19.5) -- (36.25,12);
			\draw [ color={rgb,255:red,0; green,199; blue,252}, line width=1pt, short] (36.25,12) -- (36.25,8.25);
			\draw [line width=1pt, dashed] (11.25,19.5) -- (22.5,19.5);
			\draw [line width=1pt, dashed] (27.5,19.5) -- (38.75,19.5);
			\draw [line width=1pt, dashed] (22.5,19.5) -- (22.5,8.25);
			\draw [line width=1pt, dashed] (27.5,19.5) -- (27.5,8.25);
			\draw [line width=1pt, dashed] (11.25,12) -- (38.75,12);
			\node [font=\LARGE] at (38.75,7.5) {Zeit};
			\node [font=\LARGE] at (19.25,7.75) {t};
			\node [font=\huge] at (32.5,7.75) {t};
			\node [font=\huge] at (8.75,12.25) {f};
			\node [font=\huge] at (9.75,19.75) {f};
			\node [font=\LARGE] at (9.75,22) {Frequenz};
			\node [font=\normalsize] at (19.75,7.5) {acc};
			\node [font=\normalsize] at (33,7.5) {dec};
			\node [font=\normalsize] at (10,12) {Start/Stopp};
			\node [font=\normalsize] at (10.25,19.5) {max};
			\draw [ color={rgb,255:red,0; green,199; blue,252}, line width=1pt, short] (17.5,12) -- (17.5,8.25);
			\draw [line width=1pt, ->, >=Stealth] (11.25,8.25) -- (38.75,8.25);
		\end{circuitikz}
	}%
	\caption{Trapezförmiges Geschwindigkeitsprofil \cite{FaulhaberDriveSystems.2020}} \label{Trapezförmiges Geschwindigkeitsprofil}
\end{figure}
{\textbf{Fehlerbild: Motor läuft nicht an} (siehe Ursachen und Lösungen aus Tabelle \ref{MOTLNA} \nameref{MOTLNA})% noch eine verlinkung machen ! 

{\textbf{Fehlerbild: Motor beendet die Beschleunigungsrampe nicht}
\begin{center}
	\fontsize{8}{10}\selectfont
	\begin{tabularx}{\textwidth}{|X|X|X|X|X|X|}
		\hline 
		\textbf{Ursachen} & \textbf{Lösungen} \\ \hline

  		Motor bleibt bei der Resonanzfrequenz hängen & 
  		\begin{itemize} 
  		\item {Beschleunigung erhöhen, um die Resonanzfrequenz schneller zu durchlaufen} 
  		\item {Start-Stopp Frequenz über dem Resonanzpunkt wählen} 
  		\item {Halbschritt- oder Mikroschrittbetrieb verwenden} 
  		\item {Mechanische Dämpfung vorsehen} 
  		\end{itemize}	 \\	\hline
  		Falsche Einstellung von Versorgungsspannung oder Strom zu gering & 
  		\begin{itemize}
  		\item {Spannung oder Strom erhöhen} 
  		\item {Motor mit geringerer Impedanz testen} 
  		\item {Stromregelung verwenden}
  		\end{itemize}\\	\hline
		Maximaldrehzahl zu hoch & 
		\begin{itemize}
		\item {Maximaldrehzahl reduzieren} 
		\item {Beschleunigungsrampe abflachen}
		\end{itemize}\\ \hline
		Schlechte Vorgabe der Beschleunigungsrampe durch Elektronik & 
		\begin{itemize}
		\item {Anderen Controller verwenden }
		\end{itemize}\\ \hline
\end{tabularx}
	\captionof{table}{Ursachen und Lösungen: Motor beendet die Beschleunigungsrampe nicht \cite{FaulhaberDriveSystems.2020}}	
\end{center}

{\textbf{Fehlerbild: Motor beschleunigt bis zur Enddrehzahl und bleibt stehen, sobald eine konstante Drehzahl erreicht ist.}
\begin{center}
	\fontsize{8}{10}\selectfont
	\begin{tabularx}{\textwidth}{|X|X|X|X|X|X|}
		\hline 
		\textbf{Ursachen} & \textbf{Lösungen} \\ \hline
		
		Motor wird an Leistungsgrenze betrieben und bleibt stehen aufgrund zu hoher Beschleunigung & 
		\begin{itemize} 
			\item {Ruckeln verringern durch geringere Beschleunigungsrate oder durch unterschiedliche Beschleunigungsrampen, erst steil dann flacher} 
			\item {Drehmoment erhöhen} 
			\item {Motor im Mikroschrittbetrieb betreiben} 
			\item {Mechanische Dämpfung vorsehen} 
		\end{itemize}	 \\	\hline
	\end{tabularx}
			\captionof{table}{Ursachen und Lösungen: Motor beschleunigt bis zur Enddrehzahl und bleibt stehen, sobald eine konstante Drehzahl erreicht ist. \cite{FaulhaberDriveSystems.2020}}
\end{center}



\section{Externe Ereignisse}
\subsection{Lastrückkopplung}
\textit{"Manchmal wird der vom Motor angetriebene Mechanismus/Last während der Bewegung „aufgezogen“ und gibt diese Energie wieder an den Motor zurück, wenn die Ströme ausgeschaltet werden. Der Mechanismus könnte z.B. ein Untersetzungsgetriebe sein."}\cite{FaulhaberDriveSystems.2020} 
Wird diese Energie an den Motor zurück geleitet, kann es passieren, dass der Motor sich um einen Winkel verdreht, der mehr als einem Schritt entspricht. Dabei kann es passieren, dass der Motor nicht ausreichend Drehmoment entwickelt und nicht oder erst nach 4 Vollschritten anläuft. \cite{FaulhaberDriveSystems.2020}

\textbf{Lösungen:}
\begin{itemize}
	\item Die Kommutierung so programmieren, dass Wert und Polarität vor dem Abschalten gespeichert und beim Wiedereinschalten verwenden werden kann. 
	\item Nicht vollständig den Strom abschalten, sondern bei Motorstillstand einen reduzierten \textit{Stand-By} Strom aufrechterhalten
\end{itemize}

\subsection{Erhöhung der Nutzlast mit der Zeit}
\textit{"Manchmal läuft der Motor für eine lange Zeit störungsfrei und viel später treten die ersten Schrittverluste auf. In diesem Fall ist es sehr wahrscheinlich, dass die Last, die der Motor „sieht“, sich geändert hat. Das kann auf Verschleiß der Motorlager oder ein externes Ereignis zurückzuführen sein."}\cite{FaulhaberDriveSystems.2020}

\textbf{Lösungen:}
\begin{itemize}
	\item Prüfen ob ein externes Ereignis durch Veränderung des Mechanismus vorliegt.
	\item Prüfen ob Lagerverschleiß vorhanden ist. Verwendung von Kugellager erhöhen die Lebensdauer des Motors.
	\item Verwendung von Schmiermittel um Reibung zu verhindern.
\end{itemize}
