%%%%%%%%%%%%%%%%%%%%
%
% $Beschreibung: Hier werden alle offenen Punkte, bis zum Abschluss des Projekts gesammelt $
% $Autor: ter Veen, Theilmann $
% $Datum: 09.06.2024 $
% $Pfad: DemonstratorSchrittmotor/DeveloperDoc/Contents/de/OffenePunkte.tex $
% $Version: 1 $
%
%
%%%%%%%%%%%%%%%%%%%

\chapter{Offene Punkte}

Dieses Projekt befindet sich noch in der Entwicklungsphase. Bei dem Demonstrator handelt es sich um ein Prototyp und es bietet noch an Verbesserungs- und Erweiterungspotenzial. In diesem Fall findet ausschließlich ein Ausblick zu Lern- und Demonstrationszwecken statt. 

\section{Verbesserungen}

Am offensichtlichsten ist die Überarbeitung des Gehäuses. Durch die Vibration des Motors wird häufig in die Eigenfrequenzen des Gehäuses gefahren, sodass eine starke Geräuschkulisse entsteht. Ein Spritzguss-Gehäuse mit verbesserte Konstruktion mit zum Beispiel eine Integration von elastischen Gehäusefüßen könnten diesen Mangel umgehen. Außerdem könnte weitere Einzelteile mit anderen Fertigungstechniken hergestellt werden um bessere Qualitäten der Bauteile zu erreichen.

Softwaretechnisch könnte ein verbesserter Code zu kürzeren Rechenzeiten des Prozessors und einen stabileren Lauf führen. Außerdem könnten die Bewegungsstufen besser und genau definiert werden um noch bessere Ergebnisse zu erzielen.   


\section{Erweiterung}

Die einfachste Erweiterung sind die Integration von neuen Bewegungsstufen per Software. Darüber hinaus könnte per Drehknopf eine manuelle Eingabe einer Geschwindigkeit und Beschleunigung oder ein Erreichen eines genauen Punktes hinzugefügt werden. Ein Integrieren eines Beschleunigungssensor auf dem Schlitten könnte zur genaueren Stufen führen indem das System durch den Sensor und einem Regler geregelt werden könnte. Des Weitere könnten Differenzen zwischen Eingabe und tatsächlichen Werten am OLED-Display angezeigt werden. 

\section{Anwendung in der Lehre}

Der Demonstrator kann durch die Kompakte Bauweise mobil mitgenommen und schnell aufgebaut werden. So kann der Demonstrator Studierende praktische Erfahrungen in verschieden Fachbereichen vermitteln. Die Fachrichten sind Automatisierungstechnik, Elektrotechnik, Steuerungstechnik, Maschinenelemente und viele mehr. Durch das Durchfahren der Stufen werden reale Situation von Schrittmotoren simuliert. Dies kann Einblicke in Anforderungen und Abläufe in der Industrie zeigen. 
